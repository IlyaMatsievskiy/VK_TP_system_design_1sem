\documentclass[12pt, a4paper]{article}
\usepackage[utf8]{inputenc}
\usepackage[T2A]{fontenc}
\usepackage{indentfirst, setspace}
\usepackage{tabularx, multirow}
\usepackage[normalem]{ulem}
\usepackage[style=russian]{csquotes}
\usepackage[english,russian]{babel}
\usepackage{hyperref}
\usepackage{float}
\usepackage{graphicx}
\usepackage{ragged2e}
\usepackage{caption}
\usepackage{wrapfig}
\usepackage{amsmath}
\linespread{1}
\usepackage{tikz}
\makeatletter
\def\@biblabel#1{#1. }
\makeatother
\captionsetup{labelsep=endash}


\usepackage[left=2cm,right=2cm,
    top=2cm,bottom=2cm,bindingoffset=0cm]{geometry}\begin{document}
\begin{titlepage}
     \fontsize{12}{12}\selectfont

  {\centering

   \begin{bf}



    \noindent Министерство науки и высшего образования Российской Федерации

    \noindent Федеральное государственное бюджетное образовательное учреждение высшего образования

    \noindent \enquote{Московский государственный технический университет

     \noindent имени Н.Э. Баумана

     \noindent (национальный исследовательский университет)}

    \noindent (МГТУ им. Н.Э. Баумана)

   \end{bf}
  }

  \vspace{0.4cm}

  {\setstretch{0.1}
   \noindent\rule{\textwidth}{1mm}
   \noindent\rule{\textwidth}{0.5mm}

  }

  \fontsize{14}{21}\selectfont


  \vspace{1cm}


  \begin{center}
   \begin{bf}

    \fontsize{24}{36}\selectfont
    RFC---файл

    \fontsize{20}{30}\selectfont
    Запуск демоверсии игры на сайте Steam


   \end{bf}
  \end{center}

  \fontsize{14}{21}\selectfont
  \vspace{10cm}


  \noindent\begin{tabularx}{\textwidth}{ X >{\centering}p{4cm} p{1cm} c }
   Дата создания: 25.10.2024\\
   Авторы: Мациевский Илья, Митрофанов Матвей, Могилин Никита,\\ Поперёков Глеб\\
	Ревизия: 1.0
 
   \end{tabularx}

  \vspace{\fill}

  \begin{center}
   \it{Москва}, 2024
  \end{center}

  \thispagestyle{empty}
\end{titlepage}\newpage
\setcounter{page}{2}
\tableofcontents
\newpage
\section*{Введение}
\addcontentsline{toc}{section}{Введение}
\justifying
\textbf{Описание}: функция запуска демоверсии на 
странице игры в Steam без скачивания. Данная функция 
позволит потенциальным покупателям быстро 
ознакомиться с игрой, что может увеличить 
вероятность покупки.
\par
\textbf{Обоснование}: запуск демоверсии без 
скачивания облегчает пользователям знакомство с 
новыми играми. Это особенно полезно для игроков с 
низкой скоростью интернета или с маленьким запасом
памяти на жестком диске. Функция может уменьшить
время принятия решения о покупке игры.

\newpage
\section{Бизнес план}
\textbf{Цели проекта: } увеличить вовлеченность
пользователей за счет быстрых и доступных демоверсий.
Привлечь новых пользователей уникальной на рынке
функцией. \par
\textbf{Ключевые метрики успеха: } Увеличение количества 
пользователей, запускающих демоверсии, рост конверсии из 
демоверсий в покупки, снижение среднего времени принятия 
решения о покупке, прирост новых пользователей. \par
\textbf{Ожидаемая выгода: } увеличение продаж на 
платформе и рост пользовательской лояльности.\par
\textbf{Риски и ограничения: } большие затраты на новые
системы, такие как Steam Game Streaming Engine,
зависимость игрового опыта от интернет соединения.



\section{Требования и ограничения}
\begin{enumerate}
	\item \textbf{Функциональные требования: } 
	Возможность запуска демоверсии на платформе Steam без 
	скачивания, сбор фидбека, рекомендательная система 
	Steam.
	\item \textbf{Нефункциональные требования: } 
	минимальные задержки, высокая производительность,
	масштабируемость.
	\item \textbf{Ограничения: } интернет соединение 
	пользователя, необходимость создания сервиса для
	запуска игры.
\end{enumerate}
\newpage

\section{Архитектура}

\subsection{C1-диаграмма Контекста}
На уровне C1 Steam выступает в роли центральной платформы для запуска демоверсий игры. Взаимодействие происходит между платформой Steam, пользователем, правообладателем игры и внешними сервисами, перечисленными ниже:
\begin{itemize}
    \item \textbf{Серверы} — хранят данные игр и их демоверсий;
    \item \textbf{Движок удалённого запуска демоверсий} — осуществляет запуск демоверсий, на этапе MVP можно использовать сторонний движок;
    \item \textbf{Система аналитики и отзывов} — сбор отзывов о демоверсии и самой игре;
    \item \textbf{Система рекомендаций} — подбор игр по индивидуальным предпочтениям пользователя;
    \item \textbf{Система авторизации} — авторизация на платформе. В случае низких результатов по прибыли, но высоких по игре в демоверсии можно добавить доступ к демоверсиям по подписке или от суммы, потраченной за определенный период в магазине игр;
    \item \textbf{Облачное хранилище} — сохранение данных игры пользователя.
\end{itemize}


\begin{figure}[htbp]
    \centering
    \includegraphics[height=0.52\textheight]{HW2_SD_C1.pdf}
    \caption{C1-диаграмма}
    \label{fig:C1}
\end{figure}

\newpage
\subsection{C2-диаграмма Контейнеров}
На уровне С2 система предоставления демо-версии игры 
пользователю в Steam работает следующим образом:
\begin{enumerate}
	\item Потенциальный покупатель через Desktop-
	приложение или веб-приложение запрашивает доступ 
	к демоверсии игры. Запрос передается через 
	балансировщик соединений, который управляет 
	нагрузкой.
	\item Сервис авторизации проверяет, является ли 
	пользователь авторизованным. После успешной 
	авторизации запрос передается в обработчик 
	запросов на предоставление демо игры.
	\item Обработчик запросов проверяет возможность 
	предоставления доступа к демоверсии и отправляет 
	запрос в базу данных, где хранится информация о 
	пользователях с доступом к демо.
	\item Если доступ возможен, запрос передается в 
	сервис стриминга игры, который обеспечивает 
	запуск демоверсии без загрузки на устройство 
	пользователя.
	\item Система аналитики и отзывов собирает 
	данные о взаимодействии пользователя с 
	демоверсией и предоставляет возможность оставить 
	отзыв.
	\item Система рекомендаций использует собранные 
	данные для персонализированных рекомендаций игр 
	пользователю.
	\item Внутренние системы взаимодействуют с 
	внешними с помощью сервиса API и балансировщика 
	соединений.
\end{enumerate}


\begin{figure}[htbp]
    \centering
    \includegraphics[height=0.43\textheight]{HW2_SD_C2.pdf} 
    \caption{C2-диаграмма}
    \label{fig:C2}
\end{figure}

\newpage
\subsection{C3-диаграмма Компонентов}
На уровне С3 контейнер «обработчик запросов» 
выглядит следующим образом:
\begin{enumerate}
	\item Запрос на предоставление доступа к демо 
	игры через балансировщик попадает на компонен 
	CheckAccess component, который отправляет запрос 
	в базу данных пользователей, которым 
	предоставлен доступ к демо игре.
	\item Если у пользователя есть доступ к демо 
	игры, то формируется запрос на предоставление 
	доступа пользователю к серверам Steam’а, который 
	в свою очередь попадает в брокер сообщений.
	\item SteamRequest component забирает из очереди 
	запрос на предоставление доступа и отправляет 
	его не сервера Steam’а, после чего получает 
	ответ о том, что доступ выдан.
	\item После выдачи доступа запрос передается в 
	сервис стриминга игры, который обеспечивает 
	запуск демоверсии без загрузки на устройство 
	пользователя.
	\item После выдачи доступа запрос передается в 
	API Service для взаимодействия с внешними 
	системами Steam’а.
	\item LoggingService логирует каждый вызов 
	компонента CheckAccess component и записывает 
	логи в Log Server.
\end{enumerate}


\begin{figure}[htbp]
    \centering
    \includegraphics[height=0.45\textheight]{HW2_SD_C3.pdf} 
    \caption{C3-диаграмма}
    \label{fig:C3}
\end{figure}
\newpage

\subsection{Entity-Relationship Diagram}
\begin{figure}[htbp]
    \centering
    \includegraphics[height=0.43\textheight]{HW2_SD_ER.pdf} 
    \caption{ER-диаграмма}
    \label{fig:ER}
\end{figure}
\newpage
\section{Sequence-диаграмма}


\begin{figure}[htbp]
    \centering
    \includegraphics[height=0.53\textheight]{HW2_SD_Sequence.pdf} 
    \caption{Sequence-диаграмма}
    \label{fig:Sequence}
\end{figure}
\newpage


\section{MVP}
\textbf{Включено в MVP:}
\begin{itemize}
	\item Возможность запускать демо-версию игры на веб-
	сайте без необходимости загрузки.
	\item Поддержка основного стримингового движка с 
	оптимизацией под стандартное разрешение (например, 
	720p).
	\item Отслеживание конверсии перехода от демоверсии к 
	покупке.
\end{itemize}
\par
\textbf{Не включено в MVP:}
\begin{itemize}
	\item Возможность настройки графики или других опций 
	внутри демо-версии.
\end{itemize}

\section{Roadmap}
\begin{itemize}
	\item \textbf{Этап 1: } Разработка и тестирование
	стримингового движка.
	\item \textbf{Этап 2: } Интеграция со страницей игры.
	\item \textbf{Этап 2: } Добавление рекомендательной 
	системы, системы аналитики и отзывов.
\end{itemize}

\section{Uses Cases}
\begin{itemize}
	\item \textbf{Сценарий 1: } Неавторизованный 
	пользователь пытается запустить демоверсию. Система
	просит авторизоваться.
	\item \textbf{Сценарий 2: } Авторизованный 
	пользователь пытается запустить демоверсию.
	Демоверсия запускается.
	\item \textbf{Сценарий 3: } Демоверсия прерывается 
	из-за нестабильного соединения. Прогресс сохраняется
	в облако, пользователю предлагается проверить 
	соединение.
\end{itemize}

\section{Аналитика и метрики}
\begin{itemize}
	\item Количество запусков демоверсии.
	\item Средняя продолжительность сессий.
	\item Конверсия между демоверсией и покупкой
	полной версии игры.
	\item Число сбоев, связанных с нестабильным 
	интернет соединением.
	\item Количество сбоев по другим причинам.
	\item Приток новых пользователей.
\end{itemize}
\newpage

\section{Стратегия отката и план B}
\begin{itemize}
	\item Временное отключение функции.
	\item Переключение на скачиваемые демоверсии.
	\item Временный переход на сторонний движок
	для запуска демоверсий.
\end{itemize}



\end{document}