\documentclass[12pt, a4paper]{article}
\usepackage[utf8]{inputenc}
\usepackage[T2A]{fontenc}
\usepackage{indentfirst, setspace}
\usepackage{tabularx, multirow}
\usepackage[normalem]{ulem}
\usepackage[style=russian]{csquotes}
\usepackage[english,russian]{babel}
\usepackage{hyperref}
\usepackage{float}
\usepackage{graphicx}
\usepackage{ragged2e}
\usepackage{caption}
\usepackage{wrapfig}
\usepackage{amsmath}
\linespread{1}
\usepackage{tikz}
\makeatletter
\def\@biblabel#1{#1. }
\makeatother
\captionsetup{labelsep=endash}


\usepackage[left=2cm,right=2cm,
    top=2cm,bottom=2cm,bindingoffset=0cm]{geometry}\begin{document}
\begin{titlepage}
     \fontsize{12}{12}\selectfont

  {\centering

   \begin{bf}



    \noindent Министерство науки и высшего образования Российской Федерации

    \noindent Федеральное государственное бюджетное образовательное учреждение высшего образования

    \noindent \enquote{Московский государственный технический университет

     \noindent имени Н.Э. Баумана

     \noindent (национальный исследовательский университет)}

    \noindent (МГТУ им. Н.Э. Баумана)

   \end{bf}
  }

  \vspace{0.4cm}

  {\setstretch{0.1}
   \noindent\rule{\textwidth}{1mm}
   \noindent\rule{\textwidth}{0.5mm}

  }

  \fontsize{14}{21}\selectfont


  \vspace{1cm}


  \begin{center}
   \begin{bf}

    \fontsize{24}{36}\selectfont
    Постановка задачи и SMART цель.

    \fontsize{20}{30}\selectfont
    Запуск демоверсии игры на сайте Steam


   \end{bf}
  \end{center}

  \fontsize{14}{21}\selectfont
  \vspace{10cm}


  \noindent\begin{tabularx}{\textwidth}{ X >{\centering}p{4cm} p{1cm} c }
   Дата создания: 16.11.2024\\
   Авторы: Мациевский Илья, Митрофанов Матвей, Могилин Никита,\\ Поперёков Глеб\\
	Ревизия: 1.0
 
   \end{tabularx}

  \vspace{\fill}

  \begin{center}
   \it{Москва}, 2024
  \end{center}

  \thispagestyle{empty}
\end{titlepage}\newpage
\setcounter{page}{2}
\tableofcontents
\newpage
\section{Постановка задачи по проблеме}
\textbf{Проблема: } сложность использования 
демоверсий игр.\par
\textbf{Описание проблемы: } Текущий процесс 
ознакомления с играми в Steam через демоверсии 
требует их предварительной загрузки на устройство 
пользователя. Это увеличивает время на принятие 
решения о покупке, снижает вовлеченность и может 
оттолкнуть часть потенциальных покупателей, особенно 
тех, кто не готов тратить ресурсы на загрузку.\par
\textbf{Решение: } добавить возможность запускать 
демоверсии на странице игры в Steam.\par
\textbf{Ожидаемая выгода: } увеличение продаж на 
платформе и рост пользовательской лояльности.

\subsection{План решения}
Необходимо описать архитектуру системы.
\begin{enumerate}
	\item Проверка прав доступа: требуется 
	организовать проверку прав доступа к демоверсии, 
	проверку авторизации.
	\item Интеграция стриминга: создание 
	движка для запуска игр на платформе.
	На этапе MVP подключение стороннему движку.
	\item Балансировка и отказоустойчивость.
	\item Сбор аналитики и отзывов: Реализовать 
	модуль для сбора данных о времени игры, 
	количестве запусков, возникающих ошибках, а 
	также отзывов пользователей о демоверсии.
	\item Рекомендательная система: Интегрировать 
	данные из демо-версий в существующую систему 
	рекомендаций для повышения персонализации 
	контента.
\end{enumerate}
\newpage
\section{Постановка задачи по результату}


\newpage
\section*{Цель}
Создать систему запуска демо-версии игры в Steam, благодаря которой пользователи смогут изучить игру перед покупкой.

\section*{MVP (Minimum Viable Product)}
\begin{itemize}
    \item \textbf{Основная функция:} Предоставление удаленного доступа к демо-версии игры без необходимости её загрузки.
    \item \textbf{Целевая аудитория:} Пользователи Steam.
    \item \textbf{Приоритеты:}
    \begin{itemize}
        \item Высокий: предоставление прямого удаленного доступа к демо-версии определенной игры с её страницы, уменьшение затрат на предоставление услуги.
        \item Низкий: сбор аналитических данных пользователей, составление рекомендаций для каждого пользователя.
    \end{itemize}
\end{itemize}

\section*{Шаги разработки}
\subsection*{1. Проектирование системы}
Для визуализации архитектуры системы рекомендуется использовать C2-модель и Sequence-диаграмму. Подробнее о данных моделях написано в RFC-файле.

\subsection*{2. Разработка}
\subsubsection*{Выбор технологий:}
\begin{itemize}
    \item \textbf{Backend:} Python (Django/Flask), Java (Spring Boot), GoLang, Ruby.
    \item \textbf{База данных:} PostgreSQL/MySQL, JDBC.
    \item \textbf{Очередь задач:} Kafka.
\end{itemize}

\subsubsection*{Реализация API:}
Эндпоинты:
\begin{itemize}
    \item POST: для отправки запросов от клиента, запросов от балансировщика соединений, записи в сервис логов.
    \item GET/POST: для авторизации и получения доступа к драйверу демо-игр.
\end{itemize}

\subsubsection*{Реализация логики:}
\begin{itemize}
    \item \textbf{Балансировщик соединений:}
    \begin{itemize}
        \item Принимает и обрабатывает запросы на доступ к демо-версии.
        \item Проверяет авторизацию пользователя (идет обращение к Сервису авторизации).
        \item Возвращает клиенту соединение к Сервису стриминга.
    \end{itemize}
    \item \textbf{Сервис авторизации:}
    \begin{itemize}
        \item Получает от Балансировщика соединений запрос на проверку авторизации пользователя.
        \item Аутентификация пользователя.
        \item Возвращает Балансировщику соединений результат аутентификации.
    \end{itemize}
    \item \textbf{Обработчик запросов:}
    \begin{itemize}
        \item В случае успешной авторизации получает от Балансировщика соединений запрос на предоставление доступа к демо-версии игры.
        \item Получение данных о пользователе.
        \item Подключение к БД. Проверка доступа демо-версии для пользователя.
        \item При наличии доступа передает данные о пользователе на Сервер логов.
        \item Передает запрос на доступ к демо-версии в Сервис стриминга.
    \end{itemize}
    \item \textbf{Сервер логов:}
    \begin{itemize}
        \item Получает от Обработчика запросов данные о пользователе.
        \item Записывает данные в собственную БД.
    \end{itemize}
    \item \textbf{Сервис стриминга:}
    \begin{itemize}
        \item Получает от Обработчика запросов запрос на доступ к демо-версии игры.
        \item Подключает движок, на котором происходит воспроизведение демо-версии игры.
        \item Возвращает Балансировщику соединений соединение к движку стриминга.
    \end{itemize}
\end{itemize}

\subsection*{3. Тестирование}
\begin{itemize}
    \item \textbf{Unit-тесты:} Проверка работы отдельных компонентов (сервисов).
    \item \textbf{Интеграционные тесты:} Проверка взаимодействия компонентов.
    \item \textbf{Нагрузочное тестирование:} Проверка производительности системы при высокой нагрузке.
    \item \textbf{Тестирование на сообществе:} Выбор группы игроков (например, PUBG или Factorio) для тестирования системы и сбора обратной связи.
\end{itemize}

\subsection*{4. Поддержка и развитие}
\begin{itemize}
    \item \textbf{Мониторинг системы:} Отслеживание производительности и ошибок.
    \item \textbf{Внедрение новых функций:}
    \begin{itemize}
        \item Сбор данных с Сервера логов для формирования рекомендаций на основе пройденных демо-версий игр.
    \end{itemize}
\end{itemize}

\section*{Дополнительные рекомендации}
\begin{itemize}
    \item Документируйте свой код и API для удобства дальнейшей поддержки и развития.
    \item Используйте Git для контроля версий кода и совместной работы.
    \item Постоянно поддерживайте связь с командой для обсуждения прогресса и решения возникающих проблем.
    \item Просите коллег проверять ваш код для повышения качества (Code review).
    \item Помните о важности безопасности и масштабируемости микросервиса.
\end{itemize}

\section*{Сроки}
Дедлайн по выполнению задачи — 1,5 месяца с момента начала работы.


\newpage
\section{Пример неправильной постановки задачи по 
проблеме}
\textbf{Проблема: } отсутствие возможности играть в 
игры.  
\textbf{Описание проблемы: } Пользователи не могут 
играть в игры, если у них нет доступа к полным 
версиям, а демоверсии требуют слишком много шагов 
для запуска. Это мешает людям знакомиться с играми, 
и в результате платформа теряет клиентов.  

\textbf{Решение: } добавить возможность играть в 
полные версии игр бесплатно перед покупкой.  

\textbf{Ожидаемая выгода: } увеличение времени, 
которое пользователи проводят на платформе, и рост 
числа скачиваний игр.  

\subsection{План решения}  
Необходимо описать архитектуру системы.  
\begin{enumerate}  
	\item Проверка прав доступа не требуется, 
	доступ к играм будет открыт для всех.  
	\item Интеграция стриминга: разработать 
	собственный стриминговый движок сразу, без 
	использования сторонних решений, и сделать его 
	доступным для всех игр на платформе.  
	\item Сбор аналитики и отзывов: собрать только 
	отзывы пользователей.  
	\item Рекомендательная система: не требуется, 
	пользователи сами найдут игры, которые 
	хотят попробовать.  
\end{enumerate}  
\newpage
\section{Пример неправильной постановки задачи по 
результату}


\newpage
\section{Постановка задачи по SMART}
Постановка задачи по системе SMART должна удовлетворять пяти критериям:
\begin{itemize}
	\item Spicific -- конкретная;
	\item Measurable -- измеримая;
	\item Achievable -- достижимая;
	\item Relevant -- актуальная;
	\item Time bound -- ограниченная во времени.
\end{itemize}

\subsection{Конкретная (Specific)}

Разработать, тестировать и внедрить систему запуска демо-версий игры в Steam. Система должна быть интегрирована со страницей игры, предоставлять удаленный доступ к ее демо-версии без необходимости загрузки.

\subsection{Измеримая (Measurable)}
Требуемые параметры системы:
\begin{itemize}
	\item максимальное число одновременных сессий -- 100000;
	\item время приостановки работы серсива (включая запланированное техническое обслуживание) -- не более 2 часов в месяц;
	\item время отклика -- не более 200~мс;
	\item процент покрытия кода модульным тестированием -- не менее 80\%.
\end{itemize}

\subsection{Достижимая (Achievable)}
По результатам анализа опыта команды и описанных требований, был сделан вывод, что реалистичное время разработки системы -- 3 месяца.

\subsection{Значимая (Relevant)}
Ожидаемый результат от внедрения системы:
\begin{itemize}
	\item уменьшение числа возвратов игр после их покупки на 75\%, снижение нагрузки на сервис оплаты игр на 5\%;
	\item улучшение пользовательского опыта при покупке игр;
	\item привлечение большей аудитории к новым играм.
\end{itemize}

\subsection{Ограниченная во времени (Time bound)}
Проект ограничен следующими сроками:
\begin{itemize}
	\item начало работы над проектом: 1~декабря~2024~г.;
	\item предварительный показ проекта: 1~февраля~2025~г.;
	\item финальный показ проекта: 1~марта~2025~г;
	\item запуск работы сервиса: 1~апреля~2025~г. 
\end{itemize}

\newpage
\section{Примеры неправильной постановки задачи по SMART}
\subsection{Пример №1}
\textbf{Формулировка:} За 3 месяца разработать систему удаленного запуска демо-версий игр, обеспечить 80\% покрытия и не более 2 часов приостановки работы в месяц.

\textbf{Проблема:} Задача не поставлена конкретно, никак не описано взаимодействие с уже существующими системами.

\subsection{Пример №2}
\textbf{Формулировка:} За 3 месяца разработать систему запуска демо-версий игры в Steam. Система должна быть интегрирована со страницей игры, предоставлять удаленный доступ к ее демо-версии без необходимости загрузки. 

\textbf{Проблема:} В постановке задачи никак не описаны показатели и метрики, по которым можно признать результат успешным/удовлетворительным/неудовлетворительным.
\subsection{Пример №3}
\textbf{Формулировка:} За 2 недели разработать систему запуска демо-версий игры в Steam. Система должна быть интегрирована со страницей игры, предоставлять удаленный доступ к ее демо-версии без необходимости загрузки. Обеспечить 80\% покрытия и не более 2 часов приостановки работы в месяц.

\textbf{Проблема:} При отсутствии значительных наработок по этой сфере, срок в 2 недели не представляется реалистичным -- задача не достижима.
\subsection{Пример №4}
\textbf{Формулировка:} За 3 месяца разработать систему запуска демо-версий игры в Steam. Система должна быть развернута в собственном разделе Steam, для каждой демо-версии игры должны быть созданы страницы соответствующей игры, на которых расположены ссылки на страницу полной версии игры. Система должна предоставлять удаленный доступ к ее демо-версии без необходимости загрузки. Обеспечить 80\% покрытия и не более 2 часов приостановки работы в месяц.

\textbf{Проблема:} Во-первых, задача не является конкретной, поскольку описаны сразу две разные по своей сути задачи в рамках одной. Во-вторых, задача не актуальна, поскольку в Steam уже реализованы страницы игр, интеграцию с которыми было бы разумно использовать.
\subsection{Пример №5}
\textbf{Формулировка:} Разработать систему удаленного запуска демо-версий игр, обеспечить 80\% покрытия и не более 2 часов приостановки работы в месяц. Система должна быть интегрирована со страницей игры, предоставлять удаленный доступ к ее демо-версии без необходимости загрузки. 

\textbf{Проблема:} Сроки проекта никак не ограничены.


\end{document}